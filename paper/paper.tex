% \documentclass[ams]{U-AizuGT}
\documentclass[twocolumn, a4j]{article}
% \usepackage{pifont}
% \usepackage{graphicx}
% \usepackage{cite}

% \twocolumn[
\bibliographystyle{ieice}
\author{Ryuki Hiwada}
% \studentid{s1280076}
% \supervisor{Naohito Nakasato}
% ]
\title{Domain Specific Language for high performance computing}
\begin{document}
\maketitle
\section{Abstract}
\section{Introduction}
\subsection{Background}

天文物理学や天文学において、重力相互作用するN個の粒子の力学的な進化を数値的に計算するには、N体シミュレーションが必要になる。N体シミュレーションの相互作用を求める式はfigure1に示す。
この計算にはナイーブな実装の場合、粒子の数Nに対してO(N^2)で計算量が増加していく。したがって、高速に計算するための並列化が求められる。だが、プログラムを並列化するにはコンピュータシステム
のアーキテクチャを詳細に理解する必要がある。もし並列化コードを式と粒子のデータを記述するだけで自動的に生成することができたら、上の問題は解決する。上の解決方法を実現するために、
私たちはPKGというDomain Specific Language(DSL)を作成した。DSLとは、特定の作業や問題解決のために作られた言語のことで、
今回の場合は粒子間相互作用の計算がそれにあたる。この論文では、私たちは並列化の手法としてSIMDをDSLの実装に適応した。
なぜなら、現代の多くのCPUにSIMD命令があり,汎用性が高いためである。次のセクションではどのようにして並列化するかについて説明する
% In astrophysics and astronomy, to numerically calculate the dynamical 
% evolution of N particles interacting gravitationally, N-body simulations 
% are required. Figure 1 shows the equation for interparticle interactions 
% in N-body simulations. If the equation is naively computed, the time
% complexity of calculation of interparticle interactions is 
% \begin{math}O(N^2) \end{math}, where 
% N is the number of particles. Therefore, parallelization is required to
% speed up numerical simulations. To write a parallelized code for a 
% numerical simulation, a user needs to understand the architecture of 
% computer systems in  detail. If a parallelized code is automatically
% generated by only describing the formulas and data of the numerical
% simulation, the above problems are solved. To realize the parallelization,
% we will develop Domain Specific Language \lparen DSL \rparen, which is a 
% programming language specialized to a domain, for example, SQL and HTML.
\subsection{parallelization}
SIMDは、フリンの分類[1]の中の一つで、コンピュータ算術処理に関するものです。SIMDを用いると、算術演算を複数のデータに同時に適用することができます。この論文では、
複数の粒子を計算し並列化する。例えば、Figure 1の場合、
異なる粒子の加速度の計算は互いに独立しているため、並列化できます。これにより、例えば、粒子の変数が倍精度浮動小数点数であると仮定すると、
AVX2命令を用いる場合、図1に示すように4要素の演算を実行することができます。したがって、並列化されていないコードと比較して、計算を4倍速く加速する可能性があります。
このような並列化されたコードを粒子間相互作用の式の記述から生成するために以下のようなDSLを設計しました。
\subsection{???}


\subsection{Aim of this paper}

\section{Method}
\subsection{Overview of DSL}
私たちはDSLの名前というDSLを作成しました。この言語の処理系は粒子相互作用の式を記述したコードを読み込み、その計算を高速化するようなコードを生成することが出来る。
この言語は変数を定義と相互作用の式を記述できる。変数を定義する部分では、粒子の質量や位置、結果を保存する変数、それ以外に必要となる変数を定義できる。
式の部分では、相互作用の式に必要な四則演算やsqrt,べき乗,ベクトルのノルムを計算できるようにした。この記述からkernel関数を生成し、それを呼び出して使う。
\subsection{parallelization}


\subsection{Use Sympy for DSL development.}

\section{Conclusion}
\section{Acknowledgement}
\bibliography{biblist}
\end{document}