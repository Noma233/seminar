\documentclass[a4j]{jarticle} %ここは関係ない
\usepackage{pifont}
% \usepackage{graphicx}
\usepackage{cite}
\usepackage{mathtools}
% \usepackage{listings,jvlisting}
\usepackage{listings,jvlisting} 
\usepackage[dvipdfmx]{graphicx}
\usepackage{physics}
\usepackage{tikz}
\usepackage[utf8]{inputenc}
\usepackage{booktabs} % きれいな表を作成するためのパッケージ
\usetikzlibrary{shapes,arrows,positioning}
\bibliographystyle{ieicetr}
\usetikzlibrary{positioning, arrows.meta, shapes.geometric}
\begin{document}

\begin{table}[h]
    \centering
    \caption{表のタイトル}
    \label{tab:hogehoge}
    \begin{tabular}{|c|c|c|}
        \hline  
        50000 & 5.282461100 & 4.059262 \\
  25000 & 1.322715200 & 1.008046 \\
  10000 & 0.213696 & 0.161884 \\
  1000 & 0.002116 & 0.001713 \\
    \end{tabular}
\end{table}
\end{document}