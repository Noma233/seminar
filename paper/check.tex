\documentclass{article}
\usepackage{pifont}
% \usepackage{graphicx}
\usepackage{cite}
\usepackage{mathtools}
% \usepackage{listings,jvlisting}
% \documentclass[a4j]{jarticle} %ここは関係ない
\usepackage{listings,jvlisting} 
\usepackage[dvipdfmx]{graphicx}
\usepackage{physics}
\usepackage{tikz}
\usepackage[utf8]{inputenc}
\usepackage{booktabs} % きれいな表を作成するためのパッケージ
\usetikzlibrary{shapes,arrows,positioning}
\bibliographystyle{ieicetr}
\usetikzlibrary{positioning, arrows.meta, shapes.geometric}

\begin{document}
\begin{tabular}{|l|r|r|} \hline
  50000 & 5.282461100 & 4.059262 \\
  25000 & 1.322715200 & 1.008046 \\
  10000 & 0.213696 & 0.161884 \\
  1000 & 0.002116 & 0.001713 \\
 \end{tabular}

\begin{table}
  \centering % 表を中央揃えにする
  \caption{サンプル表} % 表のタイトル
  \label{tab:sampleTable1}
  \begin{tabular}{ccc} % ここで列の数とテキストの配置を指定
  \toprule
  N & Pyker & PIKG \\
  \midrule
  50000 & 5.282461100 & 4.059262 \\
  25000 & 1.322715200 & 1.008046 \\
  10000 & 0.213696 & 0.161884 \\
  1000 & 0.002116 & 0.001713 \\
\bottomrule
\end{tabular}
\end{table}
\begin{table}[ht]
  \centering % 表を中央揃えにする
  \caption{サンプル表} % 表のタイトル
  \label{tab:sampleTable2}
  \begin{tabular}{ccc} % ここで列の数とテキストの配置を指定
  \toprule
  項目 & 値 & 単位 \\
  \midrule
  項目1 & 100 & kg \\
  項目2 & 200 & g \\
  項目3 & 300 & mg \\
\bottomrule
\end{tabular}
\end{table}


\begin{table}[ht]
  \centering % 表を中央揃えにする
  \caption{サンプル表} % 表のタイトル
  \label{tab:sampleTable3}
  \begin{tabular}{ccc} % ここで列の数とテキストの配置を指定
  \toprule
  N & Pyker & PIKG \\
  \midrule
  50000 & 24.755255 & 22.767920 \\
  25000 & 6.136025 & 5.740028 \\
  10000 & 1.044833 & 0.901585\\
  1000 & 0.010176 & 0.009109 \\
\bottomrule
\end{tabular}
\end{table}

\begin{table}[ht]
  \centering % 表を中央揃えにする
  \caption{サンプル表} % 表のタイトル
  \label{tab:sampleTable4}
  \begin{tabular}{ccc} % ここで列の数とテキストの配置を指定
  \toprule
  N & Pyker & PIKG \\
  \midrule
  50000 & 13.796789& 10.794405\\
  25000 & 3.472925 & 2.693424\\
  10000 & 0.557972 & 0.424820\\
  1000 & 0.005405& 0.004240\\
\bottomrule
\end{tabular}
\end{table}



\begin{table}[ht]
  \centering % 表を中央揃えにする
  \caption{サンプル表} % 表のタイトル
  \label{tab:sampleTable5}
  \begin{tabular}{ccc} % ここで列の数とテキストの配置を指定
  \toprule
  N & Pyker & PIKG \\
  \midrule
  50000 & 2.476695 & 2.439798\\
  25000 & 0.617220 & 0.598798\\
  10000 & 0.094859 & 0.094860\\
  1000 & 0.000939& 0.000988\\
\bottomrule
\end{tabular}
\end{table}


\begin{table}[ht]
  \centering % 表を中央揃えにする
  \caption{サンプル表} % 表のタイトル
  \label{tab:sampleTable6}
  \begin{tabular}{ccc} % ここで列の数とテキストの配置を指定
  \toprule
  N & Pyker & PIKG \\
  \midrule
  50000 & 2.153937 & 2.194700\\
  25000 &0.542478 & 0.554280\\
  10000 & 0.085537 &0.087284\\
  1000 & 0.000869&0.000874\\
\bottomrule
\end{tabular}
\end{table}

\begin{table}[ht]
  \centering % 表を中央揃えにする
  \caption{サンプル表} % 表のタイトル
  \label{tab:sampleTable7}
  \begin{tabular}{ccc} % ここで列の数とテキストの配置を指定
  \toprule
  N & Pyker & PIKG \\
  \midrule
  50000 & 15.103493& 9.904044\\
  25000 &3.736596 & 2.480025\\
  10000 &0.543367 &0.398513\\
  1000 & 0.005392&0.003911\\
\bottomrule
\end{tabular}
\end{table}


\end{document}

