
\documentclass[ams]{U-AizuGT}
\usepackage{pifont}
\usepackage{graphicx}
\usepackage{cite}

\bibliographystyle{ieice}
\author{Ryuki Hiwada}
\studentid{s1280076}
\supervisor{Naohito Nakasato}
\title{Domain Specific Language for high performance computing}
\begin{document}
\maketitle
\section{Abstract}
\section{Introduction}
\subsection{Background}
In astrophysics and astronomy, to numerically calculate 
the dynamical evolution of N particles interacting gravitationally, 
N-body simulations are required. Figure 1 shows the equation for 
interparticle interactions in N-body simulations. If the equation 
is naively computed, the time complexity of calculation of 
interparticle interactions is \begin{math}O(N^2) \end{math}, where N is the number of
particles. Therefore, parallelization is required to speed up 
numerical simulations. To write a parallelized code for a numerical
simulation, a user needs to understand the architecture of computer
systems in detail. If a parallelized code is automatically
generated by only describing the formulas and data of the numerical
simulation, the above problems are solved.
 

\section{Conclusion}
\section{Acknowledgement}
\bibliography{biblist}

\end{document}