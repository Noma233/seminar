\documentclass{article}
\usepackage{pifont}
% \usepackage{graphicx}
\usepackage{cite}
\usepackage{mathtools}
% \usepackage{listings,jvlisting}
% \documentclass[a4j]{jarticle} %ここは関係ない
\usepackage{listings,jvlisting} 
\usepackage[dvipdfmx]{graphicx}
\usepackage{physics}
\usepackage{tikz}
\usepackage[utf8]{inputenc}
\usepackage{booktabs} % きれいな表を作成するためのパッケージ
\usetikzlibrary{shapes,arrows,positioning}
\bibliographystyle{ieicetr}
\usetikzlibrary{positioning, arrows.meta, shapes.geometric}
\begin{document}

\begin{table}[ht]
\centering
\begin{tabular}{|l|c|r|}
\hline
左寄せ & 中央揃え & 右寄せ \\
\hline
行1 & データ1 & 数値1 \\
行2 & データ2 & 数値2 \\
行3 & データ3 & 数値3 \\
\hline
\end{tabular}
\caption{テーブルの例}
\label{table:example}
\end{table}

\end{document}

